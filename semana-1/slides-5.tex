\documentclass{beamer}

\usepackage[utf8]{inputenc}

\usepackage[utf8]{inputenc}
\usepackage[T1]{fontenc}
\usepackage{amsmath}
\usepackage{bicaption}
\usepackage{colortbl}
\usepackage{graphicx}
\usepackage{hyperref}
\usepackage{listings}
\usepackage{lstautogobble}
\usepackage{multicol}
\usepackage{pgffor}
\usepackage{soul}
\usepackage{tabularx}
\usepackage{tikz}
\usepackage{url}
\usepackage{xcolor}

\lstset{upquote=true}

\setbeamercolor{emph}{fg=red}
\renewcommand<>{\emph}[1]{%
  {\usebeamercolor[fg]{emph}\only#2 #1}%
}

\DeclareCaptionFont{white}{\color{white}}
\DeclareCaptionFormat{listing}{\colorbox[cmyk]{0.43, 0.35, 0.35,0.01}{#1#2#3}}

\captionsetup[lstlisting]{format=listing, singlelinecheck=false, margin=0pt,
  textfont={white,bf,tiny}, labelformat=empty}

\graphicspath{{../img/}}

\setbeamercolor{block title}{bg=cyan, fg=white}
\setbeamercolor{block body}{bg=cyan!10}

\definecolor{codegreen}{rgb}{0,0.6,0}
\definecolor{codegray}{rgb}{0.5,0.5,0.5}
\definecolor{codepurple}{rgb}{0.58,0,0.82}
\definecolor{backcolour}{rgb}{0.95,0.95,0.92}

\lstdefinelanguage{js}{
  keywords={typeof, new, true, false, catch, function, return, null, catch, switch, var, if, in, while, do, else, case, break},
  keywordstyle=\color{blue}\bfseries,
  ndkeywords={class, export, boolean, throw, implements, import, this},
  ndkeywordstyle=\color{darkgray}\bfseries,
  identifierstyle=\color{black},
  sensitive=false,
  comment=[l]{//},
  morecomment=[s]{/*}{*/},
  commentstyle=\color{purple}\ttfamily,
  stringstyle=\color{red}\ttfamily,
  morestring=[b]',
  morestring=[b]"
}

\lstdefinestyle{code}{
  backgroundcolor=\color{backcolour},
  commentstyle=\color{codegreen},
  keywordstyle=\color{magenta},
  numberstyle=\tiny\color{codegray},
  stringstyle=\color{codepurple},
  basicstyle=\ttfamily\footnotesize,
  breakatwhitespace=false,
  breaklines=true,
  captionpos=t,
  keepspaces=true,
  frame=lines,
  numbers=left,
  numbersep=5pt,
  showspaces=false,
  showstringspaces=false,
  showtabs=false,
  tabsize=2,
  autogobble
}

\lstdefinestyle{normalc}{
  style=code,
  language=c
}

\lstdefinestyle{ssc}{
 style=code,
 language=c,
 basicstyle=\ttfamily\scriptsize,
}

\lstdefinestyle{tinyc}{
  style=code,
  language=c,
  basicstyle=\ttfamily\tiny,
}

\lstdefinestyle{html}{
  style=code,
  language=html,
  basicstyle=\ttfamily,
}


\lstdefinestyle{normalbash}{
  style=code,
  language=bash
}

\lstdefinestyle{tinybash}{
  style=code,
  language=bash,
  basicstyle=\ttfamily\tiny,
}


\definecolor{links}{HTML}{2372CC}
\hypersetup{colorlinks,linkcolor=,urlcolor=links}

\parskip=15pt

\newcommand{\ilcode}[1]{\colorbox{backcolour}{\lstinline|#1|}}



\newcommand\wider[2][3em]{%
\makebox[\linewidth][c]{%
  \begin{minipage}{\dimexpr\textwidth+#1\relax}
  \raggedright#2
  \end{minipage}%
  }%
}


\title{Introdução à Computação \\ programação estruturada}
\author{Alexandre Rademaker}
\date{}

\begin{document}

\begin{frame}
 \maketitle
\end{frame}


\begin{frame}[allowframebreaks]{Programação Estruturada}

  Um programa de computador executa uma sequência de instruções para
  realizar uma tarefa. O conjunto destas instruções e como elas são
  estruturadas é o que chamamos de \textbf{algoritmo}.

  Na programação estruturada, estas instruções podem ser classificadas
  em três tipos de estrutura:

  Sequência, Decisão e Iteração.

  As instruções de um programa podem ser representadas graficamente
  por meio fluxogramas.

  \framebreak

  \begin{columns}
    \begin{column}{.6\textwidth}
      Uma \textbf{sequência} enfileirada de instruções, que são executadas
      uma após a outra.\vspace{1cm}

      No fluxograma são representadas em caixas e a ordem de realização
      das instruções é indicada por setas interligando as caixas.
    \end{column}
    \begin{column}{.4\textwidth}
      \includegraphics[width=.5\textwidth]{fluxo-seq.png}
    \end{column}
  \end{columns}
  
  \framebreak

  \begin{columns}
    \begin{column}{.6\textwidth}
      Estruturas de \textbf{decisão} são utilizadas para realizar um
      controle e um desvio nas instruções do programa, ou seja, algumas
      ações somente serão realizadas se uma determinada condição for
      atendida.
    \end{column}
    \begin{column}{.4\textwidth}
      \includegraphics[width=\textwidth]{fluxo-decisao.png}
    \end{column}
  \end{columns}

  \framebreak

  \begin{columns}
    \begin{column}{.6\textwidth}
      Iterações (repetição ou laço) são utilizadas para realizar uma mesma
      instrução (ou sequência) diversas vezes.\vspace{1cm}
      
      A repetição pode ocorrer um número pré-estabelecido de vezes ou
      quando uma condição for satisfeita.
    \end{column}
    \begin{column}{.4\textwidth}
      \includegraphics[width=\textwidth]{fluxo-repeticao.png}
    \end{column}
  \end{columns}
  
  \framebreak

  Suponha que na aula de física você aprendeu a converter uma
  temperatura de Celsius para Fahrenheit. A fórmula usada é:

  \[
    T_f = (T_c \times \frac{9}{5}) + 32
  \]

  Em uma sequência sistemática de passos com a lógica para a
  transformação pode ser.

  \framebreak

  \vfill

  \begin{center}
    \includegraphics[height=.8\textheight]{fluxograma.png}
  \end{center}

  \framebreak

  \vfill

  \begin{center}
    \includegraphics[height=.8\textheight]{fluxo-example.png}
  \end{center}

  \framebreak

  \url{https://scratch.mit.edu}

  Uma forma de programar de forma visual, quase como construíndo
  fluxogramas.
  
\end{frame}


\end{document}

%%% Local Variables:
%%% mode: latex
%%% TeX-master: t
%%% End:
