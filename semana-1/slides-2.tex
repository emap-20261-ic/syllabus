\documentclass{beamer}

\usepackage[utf8]{inputenc}

\usepackage[utf8]{inputenc}
\usepackage[T1]{fontenc}
\usepackage{amsmath}
\usepackage{bicaption}
\usepackage{colortbl}
\usepackage{graphicx}
\usepackage{hyperref}
\usepackage{listings}
\usepackage{lstautogobble}
\usepackage{multicol}
\usepackage{pgffor}
\usepackage{soul}
\usepackage{tabularx}
\usepackage{tikz}
\usepackage{url}
\usepackage{xcolor}

\lstset{upquote=true}

\setbeamercolor{emph}{fg=red}
\renewcommand<>{\emph}[1]{%
  {\usebeamercolor[fg]{emph}\only#2 #1}%
}

\DeclareCaptionFont{white}{\color{white}}
\DeclareCaptionFormat{listing}{\colorbox[cmyk]{0.43, 0.35, 0.35,0.01}{#1#2#3}}

\captionsetup[lstlisting]{format=listing, singlelinecheck=false, margin=0pt,
  textfont={white,bf,tiny}, labelformat=empty}

\graphicspath{{../img/}}

\setbeamercolor{block title}{bg=cyan, fg=white}
\setbeamercolor{block body}{bg=cyan!10}

\definecolor{codegreen}{rgb}{0,0.6,0}
\definecolor{codegray}{rgb}{0.5,0.5,0.5}
\definecolor{codepurple}{rgb}{0.58,0,0.82}
\definecolor{backcolour}{rgb}{0.95,0.95,0.92}

\lstdefinelanguage{js}{
  keywords={typeof, new, true, false, catch, function, return, null, catch, switch, var, if, in, while, do, else, case, break},
  keywordstyle=\color{blue}\bfseries,
  ndkeywords={class, export, boolean, throw, implements, import, this},
  ndkeywordstyle=\color{darkgray}\bfseries,
  identifierstyle=\color{black},
  sensitive=false,
  comment=[l]{//},
  morecomment=[s]{/*}{*/},
  commentstyle=\color{purple}\ttfamily,
  stringstyle=\color{red}\ttfamily,
  morestring=[b]',
  morestring=[b]"
}

\lstdefinestyle{code}{
  backgroundcolor=\color{backcolour},
  commentstyle=\color{codegreen},
  keywordstyle=\color{magenta},
  numberstyle=\tiny\color{codegray},
  stringstyle=\color{codepurple},
  basicstyle=\ttfamily\footnotesize,
  breakatwhitespace=false,
  breaklines=true,
  captionpos=t,
  keepspaces=true,
  frame=lines,
  numbers=left,
  numbersep=5pt,
  showspaces=false,
  showstringspaces=false,
  showtabs=false,
  tabsize=2,
  autogobble
}

\lstdefinestyle{normalc}{
  style=code,
  language=c
}

\lstdefinestyle{ssc}{
 style=code,
 language=c,
 basicstyle=\ttfamily\scriptsize,
}

\lstdefinestyle{tinyc}{
  style=code,
  language=c,
  basicstyle=\ttfamily\tiny,
}

\lstdefinestyle{html}{
  style=code,
  language=html,
  basicstyle=\ttfamily,
}


\lstdefinestyle{normalbash}{
  style=code,
  language=bash
}

\lstdefinestyle{tinybash}{
  style=code,
  language=bash,
  basicstyle=\ttfamily\tiny,
}


\definecolor{links}{HTML}{2372CC}
\hypersetup{colorlinks,linkcolor=,urlcolor=links}

\parskip=15pt

\newcommand{\ilcode}[1]{\colorbox{backcolour}{\lstinline|#1|}}



\newcommand\wider[2][3em]{%
\makebox[\linewidth][c]{%
  \begin{minipage}{\dimexpr\textwidth+#1\relax}
  \raggedright#2
  \end{minipage}%
  }%
}


\title{Introdução à Computação \\ representação de dados no computador}
\author{Alexandre Rademaker}
\date{}

\begin{document}

\begin{frame}
 \maketitle
\end{frame}


\begin{frame}[allowframebreaks]{representação de dados}

  Circuitos eletrônicos transmitem sinais. Mas como representar
  informação com eletrecidade?

  \begin{center}
    \includegraphics[width=.7\textwidth]{computer-circuit-board.jpg}
  \end{center}

  \framebreak

  Com um único fio condutor, podemos indicar dois estados (bits):
  ligado e desligado. Sim/Não, 0/1, Verdade/Falso etc.

  Mais fios, mais bits:
  
  \includegraphics[width=.8\textwidth]{wires.jpg}
  
\end{frame}


\begin{frame}[allowframebreaks]{representação binária}

  {\Large Em um computador, tudo são bits: 0 ou 1}

  \framebreak
  
  Números decimais
  \[
    123 = 1 * 10^2 + 2 * 10^1 + 3* 10^0 = 100+20+3
  \]

  Números binários
  \[
    1101 = 1 * 2^3 + 1 * 2^2 + 0 * 2^1 + 1 * 2^0 = 8 + 4 + 0 + 1 = 13
  \]
  
  \framebreak

  \begin{center}
    \begin{tabular}{rr}
      \hline
      binário & decimal\\
      \hline
      000 & 0\\
      001 & 1\\
      010 & 2\\
      011 & 3\\
      100 & 4\\
      \ldots & \\
      111 & $1 * 2^2 + 1 * 2^1 + 1 * 2^0 =  7$\\
      \hline
    \end{tabular}
  \end{center}

  Para representar um número maior que 7? Maioria dos computadores
  usam `palavras' de 8 bits ($11111111 = 255$), um byte.

  Com 32 bits podemos representar $2^{32}$ (4 bilhões de números).
  
\end{frame}


\begin{frame}[allowframebreaks]{representação de texto}

  Um caractere pode ser representado por um byte.
  \[
    A = 65 = 01000001
  \]

  Um padrão de codificação foi definido, a tabela
  \href{https://en.wikipedia.org/wiki/ASCII}{ASCII}.

  Uma sequência de caracteres
  \[
    HI! = 01001000 01001001 00100001
  \]
  
  \framebreak

  Com 8 bits temos $2^8$ possíveis valores (0 até 255)

  Para outros caracteres como acentos temos
  \href{https://en.wikipedia.org/wiki/Unicode}{Unicode} que usa mais
  bits por caractere.
  
  Um \href{https://en.wikipedia.org/wiki/Unicode}{emoji} é apenas um
  número mapeado na tabela Unicode para uma descrição (diferentes
  empresas usam diferentes imagens)

  ``face with medical mask'' = 11110000 10011111 10011000 10110111

  \framebreak

  \includegraphics[width=.8\textwidth]{hexdump.png}
  
  \ldots não existem arquivos ``texto''!  Linha de comando? Vamos
  falar sobre isso.

\end{frame}


\begin{frame}[allowframebreaks]{representando imagens}

  Usando bits mapeamos números para cores também. Sistema mais usado é
  o \href{https://en.wikipedia.org/wiki/RGB_color_model}{RGB}

  \begin{center}
    \raisebox{-0.5\height}{\includegraphics[width=.3\textwidth]{rgb.png}}
    $\to$
    \raisebox{-0.5\height}{\includegraphics[width=.2\textwidth]{rgb-color.png}}
  \end{center}

  \framebreak

  Os pontos são pixels (3 bits) e um array de pixels compõe uma
  imagem.

  \begin{center}
    \includegraphics[width=.9\textwidth]{pixels.png}
  \end{center}

\end{frame}


\begin{frame}[allowframebreaks]{áudios, vídeos e outros}

  Vídeos são apenas sequencias de imagens como um flipbook.

  \begin{center}
    \includegraphics[width=.5\textwidth]{flipbook.jpg}
  \end{center}

  \framebreak
  
  Músicas também representadas como bits, o formato
  \href{https://en.wikipedia.org/wiki/MIDI}{MIDI} com números
  representando cada notas, duração e volume.

  Outros formatos mais sofisticados existem que permitem compressão de
  dados, combinação de conteúdos diversos (imagens, textos e áudio)
  dentre outros recursos. Veja
  \href{https://en.wikipedia.org/wiki/Netpbm}{PPM},
  \href{https://en.wikipedia.org/wiki/MP3}{MP3} e
  \href{https://en.wikipedia.org/wiki/Office_Open_XML_file_formats}{DOCX}.
  
  Diferentes empresas sugerem formatos para serem usados por seus
  programas e organizações como a
  \href{https://en.wikipedia.org/wiki/Unicode_Consortium}{Unicode
    Consortium},
  \href{https://en.wikipedia.org/wiki/Institute_of_Electrical_and_Electronics_Engineers}{IEEE},
  \href{https://www.w3.org}{W3C} tentam padronizar formatos.

\end{frame}


\begin{frame}{Observação}

  Uma observação interessante.

  1 kilobyte tem $2^{10} = 1.024$ bytes e não $10^3 = 1.000$ bytes.

  Lembre-se que computadores operam em binário, contam em potências de
  2: 2, 4, 8, 16, 32 etc. 
  
  Usamos  $2^{10} = 1024$ bytes para um kilobyte porque os computadores operam em base 2 (binário), e potências de 2 são mais fáceis de manipular eletronicamente.
  
  \url{https://en.wikipedia.org/wiki/Binary_prefix}

\end{frame}


\end{document}

%%% Local Variables:
%%% mode: latex
%%% TeX-master: t
%%% End:
