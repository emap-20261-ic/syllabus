\documentclass{beamer}

\usepackage[utf8]{inputenc}

\usepackage[utf8]{inputenc}
\usepackage[T1]{fontenc}
\usepackage{amsmath}
\usepackage{bicaption}
\usepackage{colortbl}
\usepackage{graphicx}
\usepackage{hyperref}
\usepackage{listings}
\usepackage{lstautogobble}
\usepackage{multicol}
\usepackage{pgffor}
\usepackage{soul}
\usepackage{tabularx}
\usepackage{tikz}
\usepackage{url}
\usepackage{xcolor}

\lstset{upquote=true}

\setbeamercolor{emph}{fg=red}
\renewcommand<>{\emph}[1]{%
  {\usebeamercolor[fg]{emph}\only#2 #1}%
}

\DeclareCaptionFont{white}{\color{white}}
\DeclareCaptionFormat{listing}{\colorbox[cmyk]{0.43, 0.35, 0.35,0.01}{#1#2#3}}

\captionsetup[lstlisting]{format=listing, singlelinecheck=false, margin=0pt,
  textfont={white,bf,tiny}, labelformat=empty}

\graphicspath{{../img/}}

\setbeamercolor{block title}{bg=cyan, fg=white}
\setbeamercolor{block body}{bg=cyan!10}

\definecolor{codegreen}{rgb}{0,0.6,0}
\definecolor{codegray}{rgb}{0.5,0.5,0.5}
\definecolor{codepurple}{rgb}{0.58,0,0.82}
\definecolor{backcolour}{rgb}{0.95,0.95,0.92}

\lstdefinelanguage{js}{
  keywords={typeof, new, true, false, catch, function, return, null, catch, switch, var, if, in, while, do, else, case, break},
  keywordstyle=\color{blue}\bfseries,
  ndkeywords={class, export, boolean, throw, implements, import, this},
  ndkeywordstyle=\color{darkgray}\bfseries,
  identifierstyle=\color{black},
  sensitive=false,
  comment=[l]{//},
  morecomment=[s]{/*}{*/},
  commentstyle=\color{purple}\ttfamily,
  stringstyle=\color{red}\ttfamily,
  morestring=[b]',
  morestring=[b]"
}

\lstdefinestyle{code}{
  backgroundcolor=\color{backcolour},
  commentstyle=\color{codegreen},
  keywordstyle=\color{magenta},
  numberstyle=\tiny\color{codegray},
  stringstyle=\color{codepurple},
  basicstyle=\ttfamily\footnotesize,
  breakatwhitespace=false,
  breaklines=true,
  captionpos=t,
  keepspaces=true,
  frame=lines,
  numbers=left,
  numbersep=5pt,
  showspaces=false,
  showstringspaces=false,
  showtabs=false,
  tabsize=2,
  autogobble
}

\lstdefinestyle{normalc}{
  style=code,
  language=c
}

\lstdefinestyle{ssc}{
 style=code,
 language=c,
 basicstyle=\ttfamily\scriptsize,
}

\lstdefinestyle{tinyc}{
  style=code,
  language=c,
  basicstyle=\ttfamily\tiny,
}

\lstdefinestyle{html}{
  style=code,
  language=html,
  basicstyle=\ttfamily,
}


\lstdefinestyle{normalbash}{
  style=code,
  language=bash
}

\lstdefinestyle{tinybash}{
  style=code,
  language=bash,
  basicstyle=\ttfamily\tiny,
}


\definecolor{links}{HTML}{2372CC}
\hypersetup{colorlinks,linkcolor=,urlcolor=links}

\parskip=15pt

\newcommand{\ilcode}[1]{\colorbox{backcolour}{\lstinline|#1|}}



\newcommand\wider[2][3em]{%
\makebox[\linewidth][c]{%
  \begin{minipage}{\dimexpr\textwidth+#1\relax}
  \raggedright#2
  \end{minipage}%
  }%
}


\title{Introdução à Computação \\ hardware, software e sistema operacional}
\author{Alexandre Rademaker}
\date{}

\begin{document}

\begin{frame}
 \maketitle
\end{frame}


\begin{frame}[allowframebreaks]{hardware e software}

  A CPU é a unidade de processamento central, que coordena tudo que é
  executado em um computador.

  Sequencias de comandos armazenados na memória são enviados para a
  CPU para serem executados.

  Mas um computador pode estar rodando vários programas ao mesmo
  tempo!  E todos os programas precisam estar armazenados no
  computador para começar, certo?

  \framebreak

  O \textbf{sistema operacional} é o programa que gerencia o
  computador, controlando como cada software usa cada parte do
  hardware do computador.

  \framebreak

  \begin{columns}
    \begin{column}{0.5\textwidth}
      São exemplos de sistemas operacionais: Linux, Windows, MacOS,
      iOS, Android etc.\vspace{1cm}

      Eles gerenciam todos os outros programas que rodam no computador
      e quais dispositivos eles podem usar. Nele instalamos os
      programas.
    \end{column}
    \begin{column}{0.5\textwidth}
      \includegraphics[width=.8\textwidth]{so.png}      
    \end{column}
  \end{columns}

  \framebreak

  A idéia de vários programas rodando `ao mesmo tempo' é criada pelo
  SO, que permite que cada programa em execução tenha uma fração de
  segundos da CPU por vez.

\end{frame}

\begin{frame}[allowframebreaks]{sistema operacional}

  Os sistemas operacionais modernos são acompanhados de vários
  programas quando instalados.

  Existem os \textbf{utilitários}: gerenciador de arquivos, drivers
  para impressão, gerenciador de configurações, relógio,
  \textbf{interpretador de comandos} etc.

  Existem os \textbf{aplicativos}: editores de texto, editores de
  imagem, navegadores etc.

  Também podemos instalar novos programas desde que compatíveis com o
  SO.

  \framebreak

  Os programas podem ser.

  CLI (Command Line Interface) são aqueles que operam por meio de uma
  interface texto, onde o usuário interage digitando texto e recebendo
  texto como resposta, em um terminal. 

  GUI (Graphical User Interface) são aqueles onde o usuário interage
  por meio de elementos visuais, como botões, menus, janelas e ícones.

  Neste curso vamos aprender a implementar programas CLI.

  \framebreak

  Todo sistema operacional moderno (que roda em laptops e desktops)
  oferece dois utilitários muito importantes: gerenciador de arquivos
  (geralmente GUI) e interpretador de comandos (CLI).

  \framebreak

  O \textbf{gerenciador de arquivos} é usado para manipular os
  arquivos e diretórios armazenados no \textbf{sistema de arquivos}
  mantido pelo SO e armazenado no \textbf{hard-drive} do computador.

  Exemplos: Windows Explorer, Finder do MacOS, Nautilus (Linux).

  Diferente do que ocorre no iOS? Como?

  \framebreak

  O \textbf{interpretador de comandos}, ou shell, permite ao usuário
  interagir com o sistema operacional por meio de uma interface de
  linha de comando. Ele é o responsável por ler os comandos digitados
  pelo usuário e executá-los, além de fornecer informações sobre os
  resultados.

  O shell também gerencia as \textbf{variáveis de ambiente}, que são
  valores que podem ser acessados pelos programas em execução. Além
  disso, ele pode executar \textbf{scripts}, que são arquivos com uma
  sequência de comandos que podem ser executados em lote.

  Exemplos: CMD (Windows), bash, zsh, etc.

\end{frame}


\begin{frame}[allowframebreaks,fragile]{linha de comando}

  Comandos básicos no Linux: pwd, cd, ls, cp, mv, mkdir, rm, rmdir,
  touch, cat, tail, head, less, popd, pushd, ps, history, find, grep,
  df, diff, echo, man, tar and zip, etc.

  \href{https://github.com/JulianEducation/CommandLineBasics}{Tutorial
    sobre linha de comando}.

  Nomes de arquivos \ilcode{/Users/ar/work/teste.txt} (completo, nome
  base e extensão).

  Os diretórios \ilcode{.} e \ilcode{..}. 

  Opções e argumentos, \ilcode{ls -l} ou \ilcode{rm -rf}.

  \framebreak

  Em sistemas Unix/Linux, programas CLI produzem \textbf{códigos
    núméricos de retorno} para indicar como a execução ocorreu.

  Outros programas ou scripts que chamam o programa em questão podem
  usar este código para determinar se o processo foi executado com
  êxito.

  O código 0 indica sucesso. Outros códigos comuns: 1 para erro
  genérico. 2 para erro na linha de comando. 126 para alguma permissão
  negada. 127 para comando não encontrado e qualquer um acima de 128
  para casos específicos.

  Os códigos de retorno são normalmente documentados no manual do
  programa.

  \framebreak

  Quando executamos mais de um comando, os seguintes operadores podem
  ser usados:

  \begin{tabularx}{\textwidth}{lX} \hline
    \verb|A; B| & executa A e depois B\\ \hline
    \verb|A && B| & B só será executado se A retornar zero\\ \hline
    \verb$A || B$ & B só será executado se A retornar valor diferente de zero\\ \hline
  \end{tabularx}
  
  \framebreak

  Entre um programa e o seu ambiente (interpretador de comandos)
  existem \textbf{canais de comunicação padrão}.

  Estes canais são a entrada padrão (STDIN, por onde entram os dados),
  saída padrão (STDOUT, por onde saem os dados) e saída de erros
  (STDERR, usado para informar erros).

  Geralmente estes fluxos de entrada e saída de um programa CLI são
  herdado do processo `pai', o processo que evocou o programa,
  geralmente o shell.
  
  Mas podemos \textbf{redirecionar} entrada e saídas para
  arquivos. Ex: \texttt{ls -l > lista.txt}.

  \url{https://bit.ly/423O3g9}

\end{frame}


\begin{frame}{Controle de Versão}

  Um controlador de versão é uma ferramenta de software que permite
  gerenciar as alterações feitas em arquivos de um projeto
  (normalmente um diretório) ao longo do tempo.

  O mais famoso controlador de versão atualmente é o
  \href{https://git-scm.com/doc}{Git}, ver também
  \url{https://rogerdudler.github.io/git-guide/}.

  O site GitHub \url{https://docs.github.com/pt} é o maior site para
  armazenamento de projetos opensource que usam Git.

  Para saber mais:
  \href{https://www.youtube.com/results?search_query=github+tutorial}{vídeos}
  e
  \href{https://www.coursera.org/learn/introduction-git-github}{curso}.

\end{frame}

\end{document}

%%% Local Variables:
%%% mode: latex
%%% TeX-master: t
%%% End:
