\documentclass{beamer}

\usepackage[utf8]{inputenc}

\usepackage[utf8]{inputenc}
\usepackage[T1]{fontenc}
\usepackage{amsmath}
\usepackage{bicaption}
\usepackage{colortbl}
\usepackage{graphicx}
\usepackage{hyperref}
\usepackage{listings}
\usepackage{lstautogobble}
\usepackage{multicol}
\usepackage{pgffor}
\usepackage{soul}
\usepackage{tabularx}
\usepackage{tikz}
\usepackage{url}
\usepackage{xcolor}

\lstset{upquote=true}

\setbeamercolor{emph}{fg=red}
\renewcommand<>{\emph}[1]{%
  {\usebeamercolor[fg]{emph}\only#2 #1}%
}

\DeclareCaptionFont{white}{\color{white}}
\DeclareCaptionFormat{listing}{\colorbox[cmyk]{0.43, 0.35, 0.35,0.01}{#1#2#3}}

\captionsetup[lstlisting]{format=listing, singlelinecheck=false, margin=0pt,
  textfont={white,bf,tiny}, labelformat=empty}

\graphicspath{{../img/}}

\setbeamercolor{block title}{bg=cyan, fg=white}
\setbeamercolor{block body}{bg=cyan!10}

\definecolor{codegreen}{rgb}{0,0.6,0}
\definecolor{codegray}{rgb}{0.5,0.5,0.5}
\definecolor{codepurple}{rgb}{0.58,0,0.82}
\definecolor{backcolour}{rgb}{0.95,0.95,0.92}

\lstdefinelanguage{js}{
  keywords={typeof, new, true, false, catch, function, return, null, catch, switch, var, if, in, while, do, else, case, break},
  keywordstyle=\color{blue}\bfseries,
  ndkeywords={class, export, boolean, throw, implements, import, this},
  ndkeywordstyle=\color{darkgray}\bfseries,
  identifierstyle=\color{black},
  sensitive=false,
  comment=[l]{//},
  morecomment=[s]{/*}{*/},
  commentstyle=\color{purple}\ttfamily,
  stringstyle=\color{red}\ttfamily,
  morestring=[b]',
  morestring=[b]"
}

\lstdefinestyle{code}{
  backgroundcolor=\color{backcolour},
  commentstyle=\color{codegreen},
  keywordstyle=\color{magenta},
  numberstyle=\tiny\color{codegray},
  stringstyle=\color{codepurple},
  basicstyle=\ttfamily\footnotesize,
  breakatwhitespace=false,
  breaklines=true,
  captionpos=t,
  keepspaces=true,
  frame=lines,
  numbers=left,
  numbersep=5pt,
  showspaces=false,
  showstringspaces=false,
  showtabs=false,
  tabsize=2,
  autogobble
}

\lstdefinestyle{normalc}{
  style=code,
  language=c
}

\lstdefinestyle{ssc}{
 style=code,
 language=c,
 basicstyle=\ttfamily\scriptsize,
}

\lstdefinestyle{tinyc}{
  style=code,
  language=c,
  basicstyle=\ttfamily\tiny,
}

\lstdefinestyle{html}{
  style=code,
  language=html,
  basicstyle=\ttfamily,
}


\lstdefinestyle{normalbash}{
  style=code,
  language=bash
}

\lstdefinestyle{tinybash}{
  style=code,
  language=bash,
  basicstyle=\ttfamily\tiny,
}


\definecolor{links}{HTML}{2372CC}
\hypersetup{colorlinks,linkcolor=,urlcolor=links}

\parskip=15pt

\newcommand{\ilcode}[1]{\colorbox{backcolour}{\lstinline|#1|}}



\newcommand\wider[2][3em]{%
\makebox[\linewidth][c]{%
  \begin{minipage}{\dimexpr\textwidth+#1\relax}
  \raggedright#2
  \end{minipage}%
  }%
}


\title{Introdução à Computação \\ arquitetura básica dos computadores}
\author{Alexandre Rademaker}
\date{}

\begin{document}

\begin{frame}
 \maketitle
\end{frame}


\begin{frame}[allowframebreaks]{O que é um computador}

  O que diferencia um computador de outras máquinas que inventamos?
  Um microondas? Um trator? Uma máquina de lavar louças?

  Ao invés de nos ajudar com trabalhos manuais, movendo ou manipulando
  coisas físicas, computadores nos ajudam a resolver equações,
  rastrear estrelas no céu, etc.

  Nos ajudam a manipular informações ou dados.

  \framebreak

  Os computadores processam \textbf{dados} executando uma sequência de
  instruções, nós chamamos uma sequência de instruções de
  \textbf{programa de computador}.

  Os \textbf{programadores de computador} são as pessoas que criam
  programas para serem executados por computadores.

  \framebreak

  Dentro de um computador temos cabos, circuitos, dispositivos de
  entrada (teclado, mouse) e saída (monitor), placas de memória,
  conectores, etc. São o que chamamos de \textbf{hardware}.

  \textbf{Softwares} são os programas armazenados dentro do
  computador, na memória secundária (disco). São os softwares que
  'rodam' (executam) na máquina (computador).

  Softwares podem ser jogos, planilhas, editores, editores de imagens,
  etc.

  \framebreak

  Inicialmente com muitas partes mecânicas, depois com componentes
  eletrônicos lentos (1920-30). Grandes calculadoras!

  \includegraphics[width=.6\textwidth]{old-computer.jpg}

  \framebreak

  Tarefas que todos os computadores fazem:

  \begin{description}
  \item[input] receber entrada de dados, via teclado, mouse, via
    portas conectadas a sensores etc.
  \item[storage] armazenar informações/dados
  \item[processing] processar informações transformando dados em
    outros dados
  \item[output] produzir saídas. vídeos, imagens, textos, sinais para
    controladores etc.
  \end{description}

\end{frame}


\begin{frame}[allowframebreaks]{Organização do computador}

  Praticamente, todos os computadores podem ser divididos em seis
  unidades lógicas:

  \begin{itemize}
  \item dispositivos de entrada
  \item dispositivos de saída
  \item armazenamento principal
  \item armazenamento secundário
  \item unidade de aritmética e lógica 
  \item unidade central de processamento
  \end{itemize}
  
  \framebreak

  Os dispositivos de entrada

  Obtém informações (dados e programas de computador) de dispositivos
  de entrada e coloca essas informações à disposição das outras
  unidades para o processamento.
  
  Os dispositivos de entrada mais comuns são teclados, touch screens e
  mouse. Mas podemos usar microfone, digitalizar imagens e códigos de
  barra; receber um vídeo de uma webcam; receber informações de uma
  rede; obter dados de geo-posicionamento a partir de um GPS; coletar
  informações de movimento e orientação de um acelerômetro
  (smartphones); etc

  \framebreak

  Os dispositivos de saída

  A maioria das informações enviadas para a saída de computadores é
  exibida em telas, impressas em papel ou enviada para outros
  dispositivos.
  
  Os computadores também podem gerar saída de suas informações para
  redes, entre outros.

  \framebreak

  O armazenamento principal, ou memória principal (RAM).

  Capaz de armazenar relativamente poucos dados mas com acesso rápido.

  Armazena os programas de computador durante sua execução.

  Retém informações que foram inseridas pela unidade de entrada, para
  se tornarem imediatamente disponíveis para o processamento quando
  for necessário.

  As informações na unidade de memória são voláteis, são perdidas
  quando o computador é desligado.

  \framebreak

  O armazenamento secundário. Unidades CDs, DVDs, HDs, etc.

  Tem alta capacidade de armazenamento mas o acesso é mais
  lento. Novos discos SSD são mais rápidos que HDD (partes mecânicas).
  
  As informações no armazenamento secundário são persistentes;
  preservadas quando o computador é desligado. O custo por unidade de
  armazenamento secundário é muito menor.

  \framebreak

  Unidade de aritmética e lógica (ALU) é responsável pela realização
  de cálculos, como adição, subtração, multiplicação e divisão.
  
  Contém os mecanismos de decisão que permitem ao computador, por
  exemplo, comparar dois itens da unidade de memória para determinar
  se são iguais ou não.

  Nos sistemas atuais, a ALU é usualmente implementada como uma parte
  da CPU.

  \framebreak

  Unidade Central de Processamento (CPU) coordena todas as partes do
  computador.

  Muitos computadores de hoje têm múltiplas CPUs podendo realizar
  muitas operações simultaneamente.
  
  Um processador multi-core implementa múltiplos processadores em um
  único chip de circuito integrado; dual-core, quad-core etc.

  \framebreak  

  \begin{center}
    \includegraphics[width=.9\textwidth]{arquitetura-computador.png}
  \end{center}
  
\end{frame}


\begin{frame}{}
  
  {\Large Para saber mais\ldots}

  Material inspirado em
  \href{https://youtube.com/playlist?list=PLzdnOPI1iJNcsRwJhvksEo1tJqjIqWbN-}{How
    Computer Works}.
  
\end{frame}

\begin{frame}[allowframebreaks]{circuitos}

  sinais eléctricos produzidos na entrada são processados produzindo
  sinais nas saídas. Bits devem ser combinados.

  \includegraphics[width=.8\textwidth]{combinando-bits.jpg}

  \framebreak

  Componentes básicos chamados de \textbf{portas lógicas} produzem
  novos bits a partir de bits que recebem.

  \includegraphics[width=.8\textwidth]{logic-gates.jpg}

  \framebreak

  Estes componentes podem ser combinados em circuitos. Estes
  circuitos podem ter tarefas especializadas

  \includegraphics[width=.365\textwidth]{adder-1.jpg} \,
  \includegraphics[width=.55\textwidth]{adder-2.jpg}

  \framebreak

  E podemos combinar circuitos formando circuitos mais complexos.  Um
  operador de adição para números de 8 bits é formado por operadores
  de adição de números de 1 bit.

  \includegraphics[width=.8\textwidth]{8bit-adder-2.jpg}

  \framebreak

  Um multiplicador de binários:

  \includegraphics[width=.8\textwidth]{binary-multiplier.png}

  \framebreak

  Nos antigos computadores os circuitos eram longos e demandavam
  minutos para simples operações.

  \includegraphics[width=.8\textwidth]{big-old-computer.jpg}

  \framebreak

  Mas a medida que os circuitos foram sendo miniaturizados, o
  desempenho aumentou. Sinais transmitidos à velocidade da luz em
  distâncias microscópias.

  \includegraphics[width=.8\textwidth]{micro-circuit.jpg}
  
\end{frame}


\begin{frame}[allowframebreaks]{transformando entradas em saídas}

  Quando digitamos uma letra, digamos B, no teclado, bits são enviados
  para a CPU. A CPU então calcula como desenhar o símbolo na tela.

  \framebreak

  \includegraphics[width=.8\textwidth]{input-to-cpu.jpg}

  \framebreak

  \includegraphics[width=.8\textwidth]{to-output.jpg}

\end{frame}


\end{document}


%%% Local Variables:
%%% mode: latex
%%% TeX-master: t
%%% End:
