\documentclass{beamer}

\usepackage[utf8]{inputenc}

\usepackage[utf8]{inputenc}
\usepackage[T1]{fontenc}
\usepackage{amsmath}
\usepackage{bicaption}
\usepackage{colortbl}
\usepackage{graphicx}
\usepackage{hyperref}
\usepackage{listings}
\usepackage{lstautogobble}
\usepackage{multicol}
\usepackage{pgffor}
\usepackage{soul}
\usepackage{tabularx}
\usepackage{tikz}
\usepackage{url}
\usepackage{xcolor}

\lstset{upquote=true}

\setbeamercolor{emph}{fg=red}
\renewcommand<>{\emph}[1]{%
  {\usebeamercolor[fg]{emph}\only#2 #1}%
}

\DeclareCaptionFont{white}{\color{white}}
\DeclareCaptionFormat{listing}{\colorbox[cmyk]{0.43, 0.35, 0.35,0.01}{#1#2#3}}

\captionsetup[lstlisting]{format=listing, singlelinecheck=false, margin=0pt,
  textfont={white,bf,tiny}, labelformat=empty}

\graphicspath{{../img/}}

\setbeamercolor{block title}{bg=cyan, fg=white}
\setbeamercolor{block body}{bg=cyan!10}

\definecolor{codegreen}{rgb}{0,0.6,0}
\definecolor{codegray}{rgb}{0.5,0.5,0.5}
\definecolor{codepurple}{rgb}{0.58,0,0.82}
\definecolor{backcolour}{rgb}{0.95,0.95,0.92}

\lstdefinelanguage{js}{
  keywords={typeof, new, true, false, catch, function, return, null, catch, switch, var, if, in, while, do, else, case, break},
  keywordstyle=\color{blue}\bfseries,
  ndkeywords={class, export, boolean, throw, implements, import, this},
  ndkeywordstyle=\color{darkgray}\bfseries,
  identifierstyle=\color{black},
  sensitive=false,
  comment=[l]{//},
  morecomment=[s]{/*}{*/},
  commentstyle=\color{purple}\ttfamily,
  stringstyle=\color{red}\ttfamily,
  morestring=[b]',
  morestring=[b]"
}

\lstdefinestyle{code}{
  backgroundcolor=\color{backcolour},
  commentstyle=\color{codegreen},
  keywordstyle=\color{magenta},
  numberstyle=\tiny\color{codegray},
  stringstyle=\color{codepurple},
  basicstyle=\ttfamily\footnotesize,
  breakatwhitespace=false,
  breaklines=true,
  captionpos=t,
  keepspaces=true,
  frame=lines,
  numbers=left,
  numbersep=5pt,
  showspaces=false,
  showstringspaces=false,
  showtabs=false,
  tabsize=2,
  autogobble
}

\lstdefinestyle{normalc}{
  style=code,
  language=c
}

\lstdefinestyle{ssc}{
 style=code,
 language=c,
 basicstyle=\ttfamily\scriptsize,
}

\lstdefinestyle{tinyc}{
  style=code,
  language=c,
  basicstyle=\ttfamily\tiny,
}

\lstdefinestyle{html}{
  style=code,
  language=html,
  basicstyle=\ttfamily,
}


\lstdefinestyle{normalbash}{
  style=code,
  language=bash
}

\lstdefinestyle{tinybash}{
  style=code,
  language=bash,
  basicstyle=\ttfamily\tiny,
}


\definecolor{links}{HTML}{2372CC}
\hypersetup{colorlinks,linkcolor=,urlcolor=links}

\parskip=15pt

\newcommand{\ilcode}[1]{\colorbox{backcolour}{\lstinline|#1|}}



\newcommand\wider[2][3em]{%
\makebox[\linewidth][c]{%
  \begin{minipage}{\dimexpr\textwidth+#1\relax}
  \raggedright#2
  \end{minipage}%
  }%
}


\title{Introdução à Computação \\ o que é programação}
\author{Alexandre Rademaker}
\date{}

\begin{document}

\begin{frame}
 \maketitle
\end{frame}

\begin{frame}[allowframebreaks]{o que é programação}

  \begin{columns}
    \begin{column}{.5\textwidth}
  
      Programação é resolver problemas!
      \vspace{.5cm}

      Programas recebem entradas e geram saídas!
      \vspace{.5cm}

      Por que aprender a programar? Entender o problema?

    \end{column}
    \begin{column}{.5\textwidth}
      \includegraphics[width=\textwidth]{programming.png}
    \end{column}
  \end{columns}

  \framebreak

  Considere o problema de calcular raízes quadradas. Podemos definir a
  função de raiz quadrada como

  \[
    \sqrt{x} = y \quad \text{such that} \quad y > 0 \land y^2 = x
  \]

  Uma função matemática perfeitamente legítima. Poderíamos usá-la para
  \emph{reconhecer} se um número é a raiz quadrada de outro ou para
  derivar fatos sobre raízes quadradas em geral. Por outro lado, a
  definição não descreve um procedimento, não nos diz quase nada sobre
  como \emph{encontrar} a raiz quadrada de um determinado número.

  Matemática versus computação

\end{frame}


\begin{frame}[allowframebreaks]{Linguagens e Programação}

  Um software é um conjunto de instruções para que o computador
  realize ações e tome decisões.

  Existem diversas linguagens de programação, algumas entendidas
  diretamente pelos computadores, outras precisam ser 'traduzidas'.

  As LPs são divididas em três tipos gerais:
  
  \begin{itemize}
  \item Linguagens de Máquina
  \item Linguagens Assembly
  \item Linguagens de Alto-nível
  \end{itemize}

  \framebreak
  
  Um computador entende diretamente sua própria \textbf{linguagem de
    máquina}, definida por seu projeto de hardware.
  
  São constituídas por sequências de bits (1s e 0s), instruções
  associadas a operações elementares.
  
  São dependentes de máquina: uma linguagem de máquina particular
  somente pode ser usada em um único tipo de computador.

  São de difícil compreensão por humanos.

  \framebreak

  as sequências de bits (entendidas diretamente pelo computador) são
  associadas à abreviações que representavam as operações elementares
  (mnemônicos).
  
  Esses mnemônicos formaram as bases das \textbf{linguagens assembly}.
  
  O programa tradutor para converter os programas em linguagem
  assembly para programas em linguagem de máquina chama-se Assembler.

  O código é claro para o ser humano, mas incompreensível para a
  máquina até ser traduzido.
  
  \framebreak
  
  Em assembly o programador precisa escrever muitas linhas de código
  para tarefas simples.

  \textbf{Linguagens de alto-nível} aceleraram o processo de criação
  de softwares, onde instruções mais claras representam uma ou mais
  instruções para o computador.
  
  Os compiladores traduzem programas em linguagem de alto-nível para
  programas em linguagem assembly e/ou diretamente para linguagem de
  máquina.
  
\end{frame}

\begin{frame}[allowframebreaks,fragile]{Compilação}

  \begin{lstlisting}[style=normalc,caption=minimal.c]
    #include <stdio.h>

    int main()
    {
      int x = 42;
      return 0;
    }
  \end{lstlisting}

  \framebreak

  \begin{lstlisting}[style=code]
    .file   "minimal.c"
    .text
    .p2align 4,,15
    .globl  main
    .type   main, @function
    main:
     push    ebp
     mov     ebp, esp
     and     esp, -16
     sub     esp, 16
     mov     DWORD PTR [esp+12], 42
     mov     eax, 0
     leave
     ret
  \end{lstlisting}

  \framebreak

  \begin{lstlisting}[style=code]
    55               push ebp
    89 E5            mov ebp, esp
    83 E4 F0         and esp, 0xfffffff0
    83 EC 10         sub esp, 0x10
    C7 44 24 0C 2A 00 00 00  mov DWORD PTR [esp+0xc],0x2a
    B8 00 00 00 00   mov eax,0x0
    C9               leave
    C3               ret
  \end{lstlisting}
  
\end{frame}


\begin{frame}[allowframebreaks]{Programas}

  Programação Estruturada é uma metodologia de programação constituída
  por sequências, desvios e repetições de instruções de uma linguagem
  de programação.

  \framebreak

  C e C++ são duas das linguagens mais
  \href{https://solutionshub.epam.com/blog/post/programming-language-popularity-on-github}{populares}
  para o desenvolvimento de software.

  A linguagem C foi desenvolvida por Dennis Ritchie na Bell
  Laboratories. Vide \href{https://youtu.be/G1-wse8nsxY?si=lx0ygn0QGbFg_Bz0}{C Programming Language, Brian Kernighan and Lex Fridman}.

  O uso de C em vários tipos de computadores levou a muitas variações
  da linguagem.

  \framebreak

  C99 é o último padrão ANSI para a linguagem C.

  C++, uma extensão de C, foi desenvolvida por Bjarne Stroustrup no
  início dos anos 802, no Bell Laboratories.

  C11, é o padrão atual da linguagem C++ (já existe o C14).
  
\end{frame}


\begin{frame}[allowframebreaks]{desenvolvimento em C}

  Passos comuns utilizados na criação e execução de um programa em C
  (ou qualquer outra linguagem compilada).

  Existem também as linguagens \textbf{interpretadas} como Python.

  \framebreak

  fase 1: criando um programa
  
  Esta fase consiste da edição de um arquivo com um programa editor de
  texto (notepad, vi, Emacs, etc). Importante, não processadores de
  texto como Word.

  Você digita um programa C (programa fonte) usando o editor, faz as
  correções necessárias e salva o programa em um dispositivo de
  memória secundária, por exemplo, o HD.
  
  Os nomes de arquivos dos programas fonte C terminam com a extenção
  \ilcode{.c}.

  \framebreak

  Uma IDE é um editor, que normalmente tem alguma interação com o
  compilador, que facilita o desenvolvimento de programadas.

  IDE offline: \href{https://code.visualstudio.com}{VS Code},
  \href{https://www.gnu.org/software/emacs/}{Emacs}

  IDEs online: \url{http://replit.com},
  \url{https://github.com/features/codespaces} etc.
  
  \framebreak

  Fase 2: Pré-processando 
  
  Um programa pré-processador executado automaticamente antes que a
  fase de tradução do compilador inicie.
  
  O pré-processador obedece a comandos chamados diretivas do
  pré-processador, que indicam que certas manipulações são realizadas
  no programa antes da compilação.
  
  Estas manipulações usualmente juntam arquivos para serem compilados,
  e podem realizar substituições no texto.

  \framebreak

  Fase 3: Compilando um Programa C
  
  O compilador traduz o código fonte (alto-nível) em um código de
  linguagem de máquina (código objeto).

  \framebreak

  Fase 4: Ligação (linking)
  
  Tipicamente, um programa C contém referências para funções e dados
  definidos em outros arquivos, bibliotecas.

  O código objeto produzido pelo compilador C contém referências a
  trechos definidos nas bibliotecas.

  Um ligador (linker) liga o código objeto com o código das funções
  ausentes para produzir um programa executável.

  \framebreak

  Fase 5: Carga (loading)
  
  Antes de um programa ser executado, ele deve ser primeiramente
  colocado na memória (primária).

  Isto é feito pelo carregador (loader), que toma a imagem executável
  do disco e a transfere para a memória.

  Bibliotecas podem ser também carregadas apenas neste momento.

  \framebreak

  Fase 6: Execução
  
  Finalmente, o computador, sob o controle de sua CPU, executa o
  programa, uma instrução por vez.

  A maioria das arquiteturas de computadores atuais podem executar
  várias instruções em paralelo.

  \framebreak

  \vspace{1cm}
  \begin{center}
    \includegraphics[height=.8\textheight]{fases.png}
  \end{center}

\end{frame}


\end{document}

%%% Local Variables:
%%% mode: latex
%%% TeX-master: t
%%% End:
